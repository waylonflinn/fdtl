\documentclass{article}
\usepackage{amssymb, amsmath}
\title{The Numerical Solution of a Gross-Pitaevskii Equation}
\author{Earl Waylon Flinn}
\begin{document}
\maketitle
\section{Introduction}
Here we present a solution of a Gross-Pitaevskii equation (GPE), a non-linear
partial differential equation (PDE). This equation is
used to describe the behaviour of certain substances at very low temperatures.
It is used, specifically, when those substances experience a phenomenon known
as Bose-Einstein condensation. Solution of this equation yields, in addition to
the wave function, two descriptively important quantities. These are: the
chemical potential and the energy. The chemical potential replaces the energy
in it's traditional role as the eigenvalue in the PDE. The energy is then
calculated seperately with an integral.

The solution method employed is a numerical one, as no closed form solution is
known\footnote{at least none has been revealed to this author}. For ease of
implementation we have chosen the method of finite differences. Since our
Gross-Pitaevskii equation, like the standard Schr\"odinger
equations, is an elliptic one, we have employed a multigrid-like \footnote{
We use the portion typically refered to as \emph{nested iteration}. This
technique alone was sufficient for our purposes and is the only one employed
by our algorithm.} method for expeditious solution.


Below I present the equations for describing the relavent quantities (including
our Gross-Pitaevskii equation) and the equations employed in thier solution.

\subsection{Problem Overview}
Below is a presentation of the equations to be solved.\\

The Gross-Pitaevskii equation
\begin{equation}
- \frac{1}{2}  \nabla^{2} \psi +
\frac{1}{2} \left( r^{2} + \lambda^{2} z^{2} \right) \psi +
4 \pi a (N-1) \lvert \psi \rvert^{2} \psi = 
\mu \psi
\end{equation}

The energy integral
\begin{equation}
E = N \int \int \left(
	\frac{1}{2} \lvert \nabla \psi \rvert^{2} + 
	\frac{1}{2} \left( r^{2} + \lambda^{2} z^{2} \right) \lvert \psi \rvert^{2} +
	2 \pi a (N-1) \lvert \psi \rvert^{4}
	\right) 2 \pi r \, \mathrm{d} r \, \mathrm{d}z
\end{equation}

\subsection{Solution Method Overview}
Below are the operators and resulting equations employed in the solution of the above.\\

The differential operators
\begin{equation}
\frac{\partial u}{\partial r} = \frac{u_{i+1, j} - u_{i-1, j}}{2\Delta r}
\end{equation}

\begin{equation}
\frac{\partial^{2} u}{\partial r^{2}} =
\frac{u_{i+1,j} - 2 u_{i,j} +u_{i-1,j}}{\Delta r^{2}}
\end{equation}

The general form of the descretized second order eliptic equation using the
above differential operators\footnote{Using the supplied differential operators
all second order elliptic partial differential equations can be put in this
form.}
\begin{equation}
a_{i,j} u_{i+1, j} + b_{i,j} u_{i-1, j} + c_{i,j} u_{i, j+1} d_{i,j} u_{i, j-1}
+ e_{i,j} u_{i, j} = f_{i,j}
\end{equation}
In both cases the subscripts denote the grid point $(i,j)$ at which the value of the
function or coefficient is taken. Coefficents may be constant or (more likely)
have a functional dependance on the grid point.

\section{The Problem at Hand}

The Gross-Pitaevskii equation is the result of attempting to describe the,
very special, behaviour of certain substances at very low temperature. This
equation gives a reasonably good description of the phenomenon known as
Bose-Einstein condensation.


Bose-Einstein condensation describes a phemenon in which a collection of
atoms are able to inhabit the same quantum state. This gives the collection
a unique set of properties and both requires and enables a new sort of
description. This description is, too a certain extent, provided by
the Gross-Pitaevskii theory. This theory uses a mean-field approach to
describe the interactions between atoms in the condensate and simplify
the resulting equations.


The present research is concerned with the behaviour of dilute Rubidum vapors
in anisotropic harmonic trapping potentials. We use the Gross-Pitaevskii theory to formulate
an equation which describes this scenario. The result, repeated below, is a
non-linear partial differential equation requiring numerical techniques for solution.

The Gross-Pitaevskii equation
\begin{equation}
- \frac{1}{2}  \nabla^{2} \psi +
\frac{1}{2} \left( r^{2} + \lambda^{2} z^{2} \right) \psi +
4 \pi a (N-1) \lvert \psi \rvert^{2} \psi = 
\mu \psi
\end{equation}

The wave function here has been scaled to eliminate certain mathematically cumbersome
coefficients. The coefficient $\lambda^{2}$ gives the anisotropy of the trapping potential
and is a scaled ratio of the coefficients to each variable appearing in that
potential. The constant $a$, often
reffered to as the scattering length, gives a measure of the interactivity of the atoms
in the trap and $N$ here refers to the number of atoms in the condensate. It is worthwhile
to note here that it is the term containing these values which models the interaction
between constituents of the condensate. It is also this, non-linear, term which makes
solving the equation a challenge. Lastly, the eigenvalue
$\mu$ gives the value of the chemical potential.
\section{Tools Within Reach}

The method of finite differences is a way of turning a function of a continuous
variable, one defined over a real domain, into one of a discrete
variable, defined over an integer domain. This
is done by sampling the function, or restricting it's domain, and then creating
a new function which takes on only those values.
The first step is to choose the number of points at which to
sample the original function. More points give more precise results but also require
more computation. For a two dimensional function this typically
results in the production of a grid, one set of points for each independent
variable. After choosing the number of points at which to sample the original
function a new function is created with an integer domain having that number of
elements. Below are the set of equations which make these ideas a bit more
precise. We use the variables $r$ and $z$ as our original independent
variables and the real intervals $(r_{0}, r_{1})$ and $(z_{0}, z_{1})$,
respectively, as our initial domains.

\subsection{Descritization relations}

The new domains then become:
\begin{equation}
i \in \left \{ 0, 1, 2, ... I+1 \right \}
\end{equation}
\begin{equation}
j \in \left \{ 0, 1, 2, ... J+1 \right \}
\end{equation}
where I and J are the number of points at which to sample the function, in $r$
and $z$, respectively. These are, therefore, also the number of grid points in
each of the respective variables.

The following equations then describe the way in which the function is sampled.

\begin{equation}
\Delta r = \frac{r_{1}-r_{0}}{I+1}
\end{equation}
\begin{equation}
\Delta z = \frac{z_{1}-z_{0}}{J+1}
\end{equation}
\begin{equation}
r_{i} = r_{0} + i \Delta r
\end{equation}
\begin{equation}
z_{i} = z_{0} + j \Delta z
\end{equation}
where $r_0$  and $r_1$ are, respectively, the lower and upper bounds for $r$,
and $z_0$ and $z_1$ play the analogous role for $z$.\\

Our new function is, therefore, constructed in the following way.
\begin{equation}
u(i,j) = f(r_{i}, z_{j})
\end{equation}
where $u(i,j)$ is our new integer domain function, and $f(r,z)$ is our original
real domain function. We also employ the shorthand $u_{i,j}$ to stand for $u(i,j)$.

\subsection{Relaxation Techniques}
Since our PDE is an elliptic one\footnote{The type of equation found in boundary value
problems} we use a technique known as relaxation. This technique is an iterative one
which updates the value of our new function at each point based on the values of that
function at surrounding points. This is done for each point, except those on the
boundary, until all points are updated. This process is then repeated until the solution
is deemed to have converged. Convergence of this method takes place for a wide
range of equations and thier descritizations\footnote{More on this can be found
in the given references.}
We employ this method by using the differential operators given above to put
our second order partial
differential equation into a general form\footnote{also given above} which
relates the value of our function at each point to those at each of it's
neighboring points. I repeat those equations here, with the latter solved for
$u_{i,j}$ to highlight the relation between it and it's neighbors, for clarity.

The differential operators
\begin{equation}
\frac{\partial u}{\partial r} = \frac{u_{i+1, j} - u_{i-1, j}}{2\Delta r}
\end{equation}

\begin{equation}
\frac{\partial^{2} u}{\partial r^{2}} =
\frac{u_{i+1,j} - 2 u_{i,j} +u_{i-1,j}}{\Delta r^{2}}
\end{equation}

The general form of a descritized elliptic partial differential equation.

\begin{equation}
 e_{i,j} u_{i, j} = a_{i,j} u_{i+1, j} + b_{i,j} u_{i-1, j} +
 c_{i,j} u_{i, j+1} + d_{i,j} u_{i, j-1} +f_{i,j}
\end{equation}

We then use this equation to update each point on the grid, repeating in an
interative fashion, until our solution has sufficiently converged. Convergence
is usually tested by examining the extent to which the solution satisfies the
given descritized equation at each point.

It can easily be shown that each iteration is of
order $O(I\times J)$, where I and J are the number of grid points in each
variable, and the order of the entire calculation is then some function
of $I\times J$, which depends on the relaxation method employed, several
of which are available. The speed of convergence is, to a certain extent determined by how well one
 can guess an intial solution. As you can probably infer if one can guess the
 exact solution to the equation then the method converges immediately. It is
 this fact which leads us to the refinement of the relaxation technique
 presented below.

\subsection{Multigrid}
Multigrid methods are a relaxation method which use grids of varying
`fineness' to decrease computation time. `Fineness' here refers to the number
of points at which the function is sampled: more points giving a `finer' grid,
fewer points giving a more `coarse' one.
Our method begins on a coarse grid, finding a solution using traditional
relaxation methods. It then interpolates the resulting solution
to a finer grid. We then employ relaxation again but begin with the interpolated
solution as our initial guess. Since this interpolated solution is a fairly good
guess at the solution, being the solution to the problem on a slightly coarser
grid, our algorithm converges more quickly. This is then repeated with an even
finer grid, interpolating
from the final result of the previous step. This process can be repeated
indefinately on successively finer grids until the desired precision is reached
\footnote{or you get tired of waiting}. As mentioned above, this procedure is
typically referred to as \emph{nested iteration}. The complete multigrid method
, which this method is based upon, claims complexity of order $O(I\times J)$,
our method has been experimentally verified to be near this in computational
complexity.

\section{Scratch}
 As with all numerical
techniques a balance must be struck between speed and precision.
\end{document}