\documentclass{article}
\title{The Structure of the Finite Difference Template Library}
\author{Earl Waylon Flinn}
\begin{document}
\maketitle
\section{Introduction}


The Finite Difference Template Library (FDTL) is created for the purposes of quickly solving
partial differential equations (currently
only elliptic ones on rectangular grids) using the finite difference method. It is implemented in c++ using both the object
oriented (OO) paradigm and generic programming techniques. The project was initially begun to solve a specific
type of equation found in low temperature quantum mechanics (a Gross-Pitaevskii equation, a form of
Schr\"odinger's equation), but was written
in such a way as to allow the solution of a wide range of partial differential equations. It was written
to enable those with only an introductory knowledge of the finite difference method
(and not necessarily any knowledge of solution methods) to  expeditiously create efficient software
for solving an equation. It is meant, therefore, to be fast in the sense that the time required for
theoretical preparation, development and execution are all small.


This document is meant to describe the structure of the project in both the OO and generic senses and the
way in which these complementary structures relate. It is meant to do so in a way that will allow both users
and implementors to gain an understanding adequate for thier tasks.


Throughout this document I employ terminology which is standard in the object oriented and generic disciplines (such as object, inheritance,
concept and refinement). In addition to these I also use the term `category' to refer a set of related
concepts.\footnote{ See Appendix A for further discussion of my use of the term `category'.}

\section{Structure Overview}

Both the object structure and concept structure of this library can be seen as having
three basic parts. These are: \emph{problem},
\emph{solver}, and \emph{goal}. Below is a description of each and of the relations between them.

\subsection{Descriptions}

\begin{description}
\item[Problem]
A \emph{problem} represents a finite-differenced partial differential
	equation (PDE) \emph{and} a tentative solution to it.
\item[Solver]
A \emph{solver} represents a method of solving a correctly and fully
	defined \emph{problem}.
\item[Goal]
A \emph{goal} is a predicate, or test, which may be used to determine
	whether the tentative solution contained in a \emph{problem} is satisfactory.
\end{description}

\subsection{Relations}
From the above we can see that the \emph{problem} is more object-like and the \emph{solver} and
\emph{goal} are more algorithm-like (though they are all treated like objects by c++). It follows
then that the later two are likely to be things which are \emph{applied to} the former. The following
descriptions cover this relationship in more detail.
\begin{description}
\item[Solver $\rightarrow$ Problem $\leftarrow$ Goal]
A \emph{solver} is applied to a fully defined \emph{problem} until that \emph{problem}'s
  solution satisfies the \emph{goal}.
\item[Goal $\rightarrow$ Problem]
A \emph{goal} tests a \emph{problem} (and through it, it's tentative solution).
\item[Solver $\rightarrow$ Goal $\rightarrow$ Problem]
  A \emph{solver} uses a \emph{goal} to determine when a \emph{problem} is solved.
\end{description}

\section{Object Stucture}

\section{Concepts}

Assume the form of the finite differenced equation is as follows:

\begin{equation}
a_{i,j} u_{i+1, j} + b_{i,j} u_{i-1, j} + c_{i,j} u_{i, j+1} d_{i,j} u_{i, j-1} + e_{i,j} u_{i, j} = 
f_{i,j}
\end{equation}
where the subscripts denote the grid point at which the value of the function or coefficient is taken.
With the following relations
\begin{equation}
x = x_{0} + i \Delta x
\end{equation}
\begin{equation}
y = y_{0} + j \Delta y
\end{equation}
where $x_0$ and $y_0$ are the lower bounds on the $x$ and $y$ boundaries, respectively, and $\Delta x$ and
$\Delta y$ are the grid spacings in $x$ and $y$, respectively.
\subsection{Solution}
\subsubsection{Static Solution}

\begin{tabular}[!htb]{|l|l|}
\hline
Member&	Description\\
\hline
double at(int i, int j)&	returns the value at the grid point (i,j)\\
\hline
\end{tabular}

\subsubsection{Static Solution $\leftarrow$ Mutable Solution}

\begin{tabular}[!htb]{|l|l|}
\hline
Member&	Description\\
\hline
double\& u(int i, int j)&	returns a reference to the value at the grid point (i,j)\\
\hline
\end{tabular}

\subsubsection{Mutable Solution $\leftarrow$ Problem}

\begin{tabular}[!htb]{|l|p{7cm}|}
\hline
Member&	Description\\
\hline
double a(int i, int j)&	Returns the the value of the first coefficient ($a$) at the grid point (i,j)\\
\hline
double b(int i, int j)&	Returns the the value of the second coefficient ($b$) at the grid point (i,j)\\
\hline
double c(int i, int j)&	Returns the the value of the third coefficient ($c$) at the grid point (i,j)\\
\hline
double d(int i, int j)&	Returns the the value of the fourth coefficient ($d$) at the grid point (i,j)\\
\hline
double e(int i, int j)&	Returns the the value of the fifth coefficient ($e$) at the grid point (i,j)\\
\hline
double f(int i, int j)&	Returns the the value of the source term ($f$) at the grid point (i,j)\\
\hline
\end{tabular}

\subsection{Solver}
\subsubsection{Unary Function $\leftarrow$ Solver}

\begin{tabular}[!htb]{|l|p{7cm}|}
\hline
Member&	Description\\
\hline
int operator()(const Problem\& p)&	Solve the finite differenced Problem p\\
\hline
int solve(const Problem\& p)&	Solve the finite differenced Problem p\\
\hline
\end{tabular}
\subsection{Goal}
\subsubsection{Unary Function $\leftarrow$ Goal}

\subsection{Prolongation Operator}
\subsubsection{Binary Function $\leftarrow$ Prolongation Operator}

\subsection{Restriction Operator}
\subsubsection{Binary Function $\leftarrow$ Restriction Operator}

\subsection{Integrator}
\subsubsection{Unary Function $\leftarrow$ Integrator}

\subsection{Transform}
\subsubsection{Static Solution $\leftarrow$ Transform}
\section{Models}
Solver $\triangleleft$  gauss\_seidel \\
Solver $\triangleleft$  successive\_overrelaxation \\
Solver $\triangleleft$  multigrid \\
\\
Goal $\triangleleft$  residual\_norm \\
Goal $\triangleleft$  solution\_norm \\
\\
Problem $\triangleleft$  laplace \\
Problem $\triangleleft$  simple\_harmonic\_oscillator \\
Problem $\triangleleft$  gross\_pitaevskii \\
\\
Transform $\triangleleft$ gross\_pitaevskii\_energy\\

\section{Appendix A. Categories}
In addition to the standard terminology from the OO and generic schools I use the term
\emph{category} to refer to a related set of concepts. More formally, a \emph{category}
is any (disjoint) set of refinement hierarchies which link
semantically related concepts. As the typical refinement hierarchy is most easily expressed
as a graph, the category naturally contains all the concepts in the hierarchy.
I also sometimes use the name of the category to refer to one of it's constituents. This may seem a bit
confusing at first but is actually quite a natural (and common) linguistic device.


As an example consider the term `Iterator' as used in the Silicon Graphics Inc.
(SGI) documentation of the Standard Template Library (STL). `Iterator', as used there, does
not refer to an actual concept in the STL, it instead refers to a set of six related concepts.
It is consistent with both the usage in the SGI documentation and the usage in this document to create
from these six concepts, and thier refinement hierarchy, a category and to then name this category
\emph{Iterator}. This category would then include all concepts in 
the refinement hierarchy rooted at \emph{Trivial Iterator} and \emph{Output Iterator} and extending to
the concept \emph{Random Access Iterator}, which is not further refined in the STL. I believe that doing
so clarifies the role of the term `Iterator' and makes it's subsequent use more natural.

This example also serves to
illustrate the two major uses of a category's name. `Iterator' might be used, on the one hand,
to refer to the category itself, and thus to the entire set of concepts it contains, as in:
``All the concepts in \emph{Iterator} except \emph{Output Iterator} support dereferencing.''
It might also be used to refer to any of the individual concepts in this category, as in:
``Despite this fact \emph{Output Iterator} is still an \emph{Iterator}.''
\end{document}