\documentclass{article}
\usepackage{amssymb, amsmath}
\title{The Numerical Solution of a Gross-Pitaevskii Equation}
\author{Earl Waylon Flinn}
\begin{document}
\maketitle
\section{Introduction}
Here we present a solution of a Gross-Pitaevskii equation (GPE), a non-linear
partial differential equation (PDE). This equation is
used to describe the behaviour of certain substances at very low temperatures.
It is used, specifically, when those substances experience a phenomenon known
as Bose-Einstein condensation. Solution of this equation yields, in addition to
the wave function, two descriptively important quantities. These are: the
chemical potential and the energy. The chemical potential replaces the energy
in it's traditional role as the eigenvalue in the PDE. The energy is then
calculated seperately with an integral.

The solution method employed is a numerical one, as no closed form solution is
known\footnote{at least none has been revealed to this author}. For ease of
implementation we have chosen the method of finite differences. Since our
Gross-Pitaevskii equation, like the standard Schr\"odinger
equations, is an elliptic one, we have employed a multigrid-like
\footnote{Multigrid methods are a relaxation method which use grids of varying
`fineness' to decrease computation time. Our method begins on a coarse grid and
interpolates to a finer one, a procedure typically referred to as
\emph{nested iteration}. This technique alone was sufficient for our purposes
and is the only one employed by our algorithm.} method for expeditious solution.


Below I present the equations for describing the relavent quantities (including
our Gross-Pitaevskii equation) and the equations employed in thier solution.

\subsection{Problem Overview}
Below is a presentation of the equations to be solved.\\

The Gross-Pitaevskii equation
\begin{equation}
- \frac{1}{2}  \nabla^{2} \psi +
\frac{1}{2} \left( r^{2} + \lambda^{2} z^{2} \right) \psi +
4 \pi a (N-1) \lvert \psi \rvert^{2} \psi = 
\mu \psi
\end{equation}

The energy integral
\begin{equation}
E = N \int \int \left(
	\frac{1}{2} \lvert \nabla \psi \rvert^{2} + 
	\frac{1}{2} \left( r^{2} + \lambda^{2} z^{2} \right) \lvert \psi \rvert^{2} +
	2 \pi a (N-1) \lvert \psi \rvert^{4}
	\right) 2 \pi r \, \mathrm{d} r \, \mathrm{d}z
\end{equation}

\subsection{Solution Method Overview}
Below are the operators and resulting equations employed in the solution of the above.\\

Descritization relations
\begin{equation}
i \in \left \{ 1, 2, ... I+1 \right \}
\end{equation}
\begin{equation}
j \in \left \{ 1, 2, ... J+1 \right \}
\end{equation}
where I and J are the number of (internal) grid points in $r$ and
$z$, respectively.
\begin{equation}
r = r_{0} + i \Delta r
\end{equation}
\begin{equation}
z = z_{0} + j \Delta z
\end{equation}
\begin{equation}
\Delta r = \frac{r_{1}-r_{0}}{I+1}
\end{equation}
\begin{equation}
\Delta z = \frac{z_{1}-z_{0}}{J+1}
\end{equation}
where $r_0$  and $r_1$ are, respectively, the lower and upper bounds for $r$,
and $z_0$ and $z_1$ play the analogous role for $z$.\\

The differential operators
\begin{equation}
\frac{\partial u}{\partial r} = \frac{u_{i+1, j} - u_{i-1, j}}{2\Delta r}
\end{equation}

\begin{equation}
\frac{\partial^{2} u}{\partial r^{2}} =
\frac{u_{i+1,j} - 2 u_{i,j} +u_{i-1,j}}{\Delta r^{2}}
\end{equation}

The general form of the descretized second order eliptic equation using the
above differential operators\footnote{Using the supplied differential operators
all second order elliptic partial differential equations can be put in this
form.}
\begin{equation}
a_{i,j} u_{i+1, j} + b_{i,j} u_{i-1, j} + c_{i,j} u_{i, j+1} d_{i,j} u_{i, j-1}
+ e_{i,j} u_{i, j} = f_{i,j}
\end{equation}
where the subscripts denote the grid point $(i,j)$ at which the value of the
function or coefficient is taken. Coefficents may be constant or (more likely)
have a functional dependance on the grid point.

\section{Solution Method}

\section{Scratch}

\end{document}